\documentclass[]{article}
\usepackage{multicol}
\usepackage[margin=1in]{geometry}
\pagestyle{empty}
\sloppy
\title{CSCI 321, Game Journal Specifications}
\author{Geoffrey Matthews}
\begin{document}
\maketitle
\begin{itemize}
\item
Your game journal is a record of what games you played, and what you
learned from them.
\item You should record entries in your journal as soon
as possible after playing games, and not leave them off to the last
minute.
\item You are required to play at least two hours a week.
  \item Include a word count at the top of every submission. Each 
weekly  journal entry must have at least 250 words.
  \item Game journals should be simple text files (preferred for size
    and loading speed), not word or pdf.
\item
You should not record mere mechanics and trivia about the game, but
insight and analysis about what makes playing it a good or bad
experience.  Consult the notes on evaluating games, also on the
website, for parameters to consider in evaluating games.  Just as
examples, consider questions like the following:


\begin{itemize}
\item What technique did you use to analyze your experience?
\item What experience were the game designers trying for?  How well
  did they acheive it?
\item How did the game pique your interest?
\item How well were each of the four elements (aesthetics, mechanics,
  technology and story) developed in the game?  How good was the
  balance?
\item Was there a theme?  Was it reinforced in every element?
\item What was the target audience for the game?  How well did the
  game suit it?
\item What was your dominant emotion while playing?  How was it
  brought about?
\item Does the game keep your interest for a long time?  How?
\item Analyze the game's mechanics---could they have been improved?
  How? 
\item Were various elements balanced?
\item Were the puzzles fair?
\end{itemize}

\end{itemize}
Example game journal I wrote a few years ago:

\setlength{\parindent}{0in}

\bigskip
\tt

Game Journal,
Geoffrey Matthews

Entry:  Monday, Jan 9, 2012
Word count:  730 words.

Got a new smartphone last week, and so I spent some time over the
weekend playing some casual games that come free with the phone.

BEJEWELED 2: Casual game.  Played 30 minutes, approximately
1:00-1:30pm Saturday.  Demo version comes with phone, small price for
full version (probably not worth it).  This game has a very relaxed
feel, searching for three-in-a-row patterns with no time pressure
(standard version), is very relaxing.  The glowing, sparkling, jewels
give the game a rich, luxurious feeling, which goes well with the
relaxing feel.  The music is particularly annoying--it is some kind of
funky, disco-feel trash, completely jarring with the rich and relaxed
look and feel of the game.  One feature I found annoying, and again
clashing with the "relaxation" feel, is the "hint" that pops up if you
take too long looking for a match without doing anything.  I resent
the pressure to find a match before the hint comes up, and after the
hint comes up I feel insulted--"Look you idiot, move THIS one."  It's
the same feeling you get if someone comes up to your crossword puzzle
or sudoku and tells you where to fill something in--you're not
grateful for the help, you resent it.  I suppose you could turn off
the hint somewhere in the preferences, but given that the game is
supposed to be a casual game you should not have to spend time
configuring it.  Perhaps if it was more obvious how to turn the
feature off.  Casual games like this should not warrant reading
anything to figure out.  You open it, you play it, you go do something
else.

ANGRY BIRDS: Free download.  Played 2 hours or more, maybe 2:00-4:00pm
or so.  Free download version has annoying popup ads, but they are
mostly ignorable.  This game is genuinely addicting--why?  For one
thing, the pure joy in the random destruction is excellent.  For
another, the "odds" of winning with only moderate skill seem to be
about what they are in Solitaire--maybe 1 one 5 tries?  This seems to
be about right to keep you playing, not too frustrating, not too easy.
For another thing, the theme and aesthetics are charming.  The birds
and the pigs make cute noises as a constant running commentary and
don't get annoying in the first hour or two (I particularly like the
way the pigs laugh at you when you lose).  Another thing is the many
different ways to win--the physics is such that very unpredictable
things happen (kind of an anti-tetris), and you can struggle to win a
level for some time, and then when you finally win on a fluke you have
two or three birds left over.  This gives it a constant surprise
factor, wanting to see what happens next.  The skill component is
somewhat disappointing.  The slightest deviation in firing the birds
makes for radically different outcomes, and so there is a large degree
of chance in the game which can get frustrating in some levels.  I
finally gave up on a level in which (it seemed) the only way to win
was to fire birds with great precision over and over.  This seemed to
make the level "too hard", and made me resent the fact that all other
levels beyond (possibly some very enjoyable ones) were blocked until I
played this one over and over, doing essentially the same thing every
time, until I got the aiming "just exactly right".  I quit after
trying this level for a while.  When I came back to play it again the
next day, I tried this level a couple of times, but gave up again and
went back to replay some of the solved levels.  That was way more fun
than the frustrating "skill" level.  This is NOT a game of skill--but
it could be if there was a better way of getting feedback on your
aiming.  Perhaps they could add a "predictor" arc that showed you
where the bird would go when you let go.  Right now its mostly a
guess.  That does enhance the level of surprise and joy in random
destruction, but then they should not put in levels with required
skill components.  All levels should be defeatable by "luck" and
moderate skill after a few dozen tries.  The levels before this one
seemed to follow that model.



\end{document}

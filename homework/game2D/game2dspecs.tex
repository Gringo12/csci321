\documentclass{article}
\usepackage[margin=1in]{geometry}
\usepackage{hyperref}
\usepackage{graphicx}
\usepackage{multicol}
\begin{document}

\title{CSCI 321, Specifications for 2D Games}
\author{Geoffrey Matthews}

\maketitle

\begin{enumerate}

\item Game programming must be your own work.  You may consult with
  each other on ideas, artwork, {\em etc.}, and use each other as game
  testers, but you may not consult on source code.

\item {\bf All games must be original work.} You can clone an existing
  game design, but the majority of the programming must be your own.
  You can use modules and code of others, but it must be acknowledged
  both in the code and in the documentation, and the bulk of the code
  must be your own. The artwork and sounds may be original or not, but
  credit must be given for all work that is not your own.

\item All games must be accompanied by a {\bf Game Manual} in PDF format.
  The game
  manual must include:
  \begin{itemize}
\item {\bf Title page.}
  Name for the game, your name(s), student number(s), 
  class number, quarter,
  instructor's name.
\item {\bf Back story} for the game (if any).  If you have different
  kinds of NPCs (non-player characters), give each of them their
  story.
\item {\bf User's guide.}
  This includes how to play, what the objective of the game is, how
  scoring is decided, different play modes, {\em etc.}.  Most
  important: how to locate the in-game help
  screen (ususally {\tt F1}), and how to quit the game (usually
  {\tt ESC}).  Another common pattern is to have {\tt ESC} bring
  up a menu, and two of the items on the menu are always {\tt Quit}
  and {\tt Help}.
\item {\bf Module documentation.}  A brief summary of each module in your
  code, what functions and classes are found there, which modules they
  import.  Also, a brief overview of how the code fits together.
\item {\bf Cheats.}  I don't have time to solve all your puzzles, or
  acquire the skill to defeat your game.  A good game should take
  hours to finish, but I want to see everything you did in a few
  minutes.  Please provide walkthroughs for your puzzles, and
  genuine cheats (keys to press to get to the next level without
  earning it, an immortality mode, unlimited ammo, the BFG,
  {\em etc.}), so I can see your
  whole game in just a few
  minutes.
\item {\bf Acknowledgement} of credit for any artwork, sounds, or code that
  is not your own.
  Failure to acknowledge such is plagiarism and grounds for academic
  disciplinary action.  Even if the stuff you use is public domain, it
  is always nice to credit the place you got it.  Whenever you grab
  something off the internet, just save the URL with the file and list
  all these in your acknowledgements.
\item {\bf Autobiographical info} on the programmer.
  There is a wide skill range in this class in programming, and in
  game knowledge.  Games will be judged individually, and based on
  what the student brings to the game.  Some students may focus on the
  artwork, some on the game physics, some on the puzzles, some on
  trying out a completely novel game idea, {\em etc.}  Needless to
  say, grading will be very subjective.  However, I will attempt to
  discern how much effort went into the project.  To that end, it is
  not inappropriate here to give some autobiographical information in
  the Game Manual about your history, what you found challenging, what
  was easy, how this differs from games you've done in the past, {\em
    etc.}
  \end{itemize}

\item All games must have {\bf in-game documentation.}  Usually this
takes the form of a help screen, traditionally accessed by pressing
{\tt F1} or by choosing a {\tt Help} menu item.  The help screen
should give much of what is in the user's guide, but more succinctly.

\newpage

\item{\bf  Grading criteria.}  Notwithstanding the range of games I expect to
see, here are some of the things I look for in evaluating a game
(based on John Laird's criteria).  
%\begin{multicols}{2}
  \begin{enumerate}
\item{\bf Is the game functional? }
\item{\bf Manual and in-game documentation. }
      Things to consider:
      \begin{itemize}
  \item Is the  documentation clear?
     \item Is it entertaining and inviting?
     \item Is there a back story to the game?
     \item Does the game have in game instructions on keys and controls?
     \end{itemize}
  \item {\bf Non-trivial implementation. }
      Things to consider:
     \begin{itemize}
     \item Sound effects
     \item Music
     \item Different difficulty levels
     \item Multiple opponents with different behavior
     \item Multiple levels
     \item Tutorial level
     \item Complex interactions between player and enemies
     \item Complex properties of game pieces (health, shields, ...)
     \item Two player game mode
     \item Graphics unusual and engaging
     \item Physics complex
     \item Universe bigger than screen
     \item Deep story
     \item Network support
     \end{itemize}
  \item {\bf Bugs/Design Flaws.}
      Things to consider:
     \begin{itemize}
     \item Game crashes
     \item Game locks up
     \item Long load time
     \item Bad controls
     \item No ability to stay in game after one turn
     \item Only one life
     \item Difficult to tell if you're making progress
     \item Collision detection problems (walking through walls, getting
     stuck in walls, {\em etc.})
     \end{itemize}
  \item {\bf  Game Play.}
     Things to consider:
    \begin{itemize}
    \item Impossible to win
    \item Too easy to win
    \item Just a move and shoot game
    \item Rate of feedback and achievement not good
    \item Only single level of goals
    \item Not fun
    \item Good mechanics for gameplay
    \item Originality of game
    \item Goals are well integrated and are appropriately rewarding
    \item Feedback appropriately informs players of what is important
    \item Dynamic difficulty adjustment
    \item Judge gets hooked on playing the game
    \item Judge shouts, screams, or laughs out loud while playing
    \end{itemize}
      \end{enumerate}

%\end{multicols}
  \end{enumerate}
\end{document}
\end

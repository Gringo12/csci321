\documentclass{beamer}
%\documentclass[handout]{beamer}

\usepackage{rotating}

\title{How to Evaluate Games}
\subtitle{Some Ideas from\\ {\em The Art of Game Design, a
  Book of Lenses}\\ Jesse Schell}
%\author{}
%\author[R. Rostamian]{Rouben Rostamian}
%\institute[WWU]{
%  Department of Computer Science \\
%  Western Washington University \\
%  Bellingham, WA 98225
%}

\begin{document}

%----------- titlepage ----------------------------------------------%
\begin{frame}[plain]
  \titlepage
\end{frame}

%----------- slide --------------------------------------------------%
\begin{frame}
  \frametitle{Playing a Game is an {\em Experience}}

\begin{itemize}
\item The experience is the most important part of a game.
\item To evaluate a game, evaluate your experience of it.\pause
  \item How does it feel to play the game?
  \item Exciting, frustrating, intriguing, frightening, ...
  \item Why do you keep playing?  Why did you start?
\item What could be improved to make the experience richer?
\item What games give you similar experiences?
\item What games are completely different?
\item What experiences would you like to see in this game?
\end{itemize}

\end{frame}

%----------- slide --------------------------------------------------%
\begin{frame}
  \frametitle{Looking at Experience}

\begin{itemize}

\item How can you analyze your feelings while you're still having
  them?\pause
\item Look at your memories\pause
\item Play the same game twice\pause
\item Sneak glances\pause
\item Observe silently, like a Zen master

\end{itemize}

\end{frame}

%----------- slide --------------------------------------------------%
\begin{frame}
  \frametitle{What is Play?}

\begin{itemize}
  \item Play is the aimless expenditure of exuberant energy.\pause
\item Play refers to those activities which are accompanied by a state
  of comparative pleasure, exhilaration, power, and the feeling of
  self-initiative.\pause
\item Play is free movement within a more rigid structure.\pause
\item Play is whatever is done spontaneously and for its own sake.\pause
\item Play is manipulation that indulges curiosity.
\end{itemize}

\end{frame}

%----------- slide --------------------------------------------------%
\begin{frame}
  \frametitle{Curiosity:  Play Answers Questions}

\begin{itemize}
  \item What happens when I click this?
\item Can I beat this team?
\item What can I make with this clay?
\item How many freethrows can I make in a row?
\item How do I finish this level?
\item What new monsters will I find next?
\item How do I use this weapon?
\end{itemize}

When evaluating a game, ask yourself what questions it raises, and how
satisfactorily it answers them.

\end{frame}

%----------- slide --------------------------------------------------%
\begin{frame}
  \frametitle{What is a Game?}

\begin{itemize}
  \item Games are an exercise of voluntary control systems, in which
    there is an contest between powers, confined by rules in order to
    produce a disequilibrial outcome.\pause
\item A game is an interactive structure of endogenous meaning that
  requires players to struggle toward a goal.\pause
\item A game is a closed, formal system, that engages players in
  structured conflict, and resolves in an unequal outcome.\pause
\item A game is a problem-solving activity, approached with a playful
  attitude. 
\end{itemize}

\end{frame}

%----------- slide --------------------------------------------------%
\begin{frame}
  \frametitle{Game Qualities}

\begin{itemize}
  \item Games are entered willfully
\item Games have goals
\item Games have conflict
\item Games have rules
\item Games can be won and lost
\item Games are interactive
\item Games have challenge
\item Games can create their own internal value
\item Games engage players
\item Games are closed, formal systems.

\end{itemize}

When you evaluate a game, look at how well or poorly it achieves these
qualities.  What could be done to improve them?

\end{frame}

%----------- slide --------------------------------------------------%
\begin{frame}
  \frametitle{Games Pose Problems}

\begin{itemize}
  \item Find a way to get more points than the other guys
\item Find a way to defeat a monster
\item Find a way through a maze
\item Find a way to complete this level
\item Find a way to destroy the other player.
\end{itemize}

When evaluating a game:  What problems does the game give the player?
Are there hidden problems that arise?  How does the game generate new
problems each time, so players come back?

\end{frame}

%----------- slide --------------------------------------------------%
\begin{frame}
  \frametitle{The Four Elements of Games}

\begin{picture}(200,120)
\put(100,100){\makebox(0,0){\it More visible}}
\put(100,0){\makebox(0,0){\it Less visible}}
\put(100,75){\oval(60,20)}
\put(100,75){\makebox(0,0){Aesthetics}}
\put(100,25){\oval(60,20)}
\put(100,25){\makebox(0,0){Technology}}
\put(50,50){\oval(60,20)}
\put(50,50){\makebox(0,0){Mechanics}}
\put(150,50){\oval(60,20)}
\put(150,50){\makebox(0,0){Story}}
\put(100,50){\vector(1,0){20}}
\put(100,50){\vector(0,1){10}}
\put(100,50){\vector(-1,0){20}}
\put(100,50){\vector(0,-1){10}}
\end{picture}

\bigskip
\begin{itemize}
\item Which areas are best?  Worst? 
\item  How could
they be bettered? 
\item Is there {\em balance} and {\em blend} between all
four?
\end{itemize}
\end{frame}

%----------- slide --------------------------------------------------%
\begin{frame}
  \frametitle{Theme}

\begin{itemize}
  \item What is the game {\em about?}\pause
\item {\em Titanic}:  Love is stronger than death.\pause
\item {\em Hercules}:  Virtue can defeat death.\pause
\item {\em Toontown online}:  Play can defeat work.\pause
  \item Does every element of the game support the theme?
\end{itemize}

\end{frame}

%----------- slide --------------------------------------------------%
\begin{frame}
  \frametitle{The Players}

\begin{itemize}
  \item What is the target audience for the game?
\item Age, sex, occupation, politics, personality,...\pause
\begin{columns}[c]
\column{0.45\textwidth}
\item Males prefer:
\begin{itemize}
\item Mastery \item Competition \item Destruction \item Spatial
  Puzzles \item Trial and Error \end{itemize}\pause
\column{0.45\textwidth}
\item Females prefer:
\begin{itemize}
\item Emotion \item Real world \item Nurturing \item Dialog and Verbal
  Puzzles \item Learning by Example
\end{itemize}
\end{columns}
\end{itemize}




\end{frame}

%----------- slide --------------------------------------------------%
\begin{frame}
  \frametitle{Game Pleasures}

\begin{itemize}
  \item Sensation
\item Fantasy
\item Narrative
\item Challenge
\item Fellowship
\item Discovery
\item Expression
\item Submission
\end{itemize}



\end{frame}

%----------- slide --------------------------------------------------%
\begin{frame}
  \frametitle{Taxonomy of Player Types}

\begin{picture}(200,100)
\put(100,100){\makebox(0,0){\bf Acting}}
\put(100,0){\makebox(0,0){\bf Interacting}}
\put(0,50){\makebox(0,0){\bf Players}}
\put(200,50){\makebox(0,0){\bf World}}
\put(100,50){\vector(1,0){75}}
\put(100,50){\vector(-1,0){75}}
\put(100,50){\vector(0,1){40}}
\put(100,50){\vector(0,-1){40}}
\pause
\put(175,75){\makebox(0,0){\it Achievers}}\pause
\put(175,25){\makebox(0,0){\it Explorers}}\pause
\put(25,25){\makebox(0,0){\it Socializers}}\pause
\put(25,75){\makebox(0,0){\it Killers}}
\end{picture}

\end{frame}

%----------- slide --------------------------------------------------%
\begin{frame}
  \frametitle{More Subtle Pleasures}

\begin{itemize}
  \item Anticipation
\item Delight in Another's Misfortune
\item Gift Giving
\item Humor
\item Possibility
\item Pride in Accomplishment
\item Purification
\item Surprise
\item Thrill
\item Triumph over Adversity
\item Wonder
\end{itemize}

\end{frame}



%----------- slide --------------------------------------------------%
\begin{frame}
  \frametitle{The Learning Curve}

\begin{picture}(200,150)
\put(0,75){\begin{turn}{90}\makebox(0,0)[b]{Skill}\end{turn}}
\put(100,0){\makebox(0,0)[t]{Time}}
\put(0,0){\vector(1,0){200}}
\put(0,0){\vector(0,1){150}}

\qbezier(0,0)(75,0)(100,75)
\qbezier(100,75)(125,150)(200,150)
\put(200,150){\vector(1,0){1}}
\end{picture}

\end{frame}


%----------- slide --------------------------------------------------%
\begin{frame}
  \frametitle{Flow}

\begin{picture}(200,150)
\put(0,75){\begin{turn}{90}\makebox(0,0)[b]{Challenges}\end{turn}}
\put(100,0){\makebox(0,0)[t]{Skills}}
\put(0,0){\vector(1,0){200}}
\put(0,0){\vector(0,1){150}}

\put(150,20){\it Boredom}
\put(20,100){\it Anxiety}
%\put(0,0){\vector(3,2){150}}
\put(80,60){\bf Flow}
\qbezier(0,0)(40,0)(40,30)
\qbezier(40,30)(40,60)(80,60)
\qbezier(80,60)(120,60)(120,90)
\qbezier(120,90)(120,120)(160,120)
\put(160,120){\vector(1,0){1}}
\end{picture}

\end{frame}

%----------- slide --------------------------------------------------%
\begin{frame}
  \frametitle{Factors Affecting Flow}
\begin{itemize}
  \item Clear goals
\item No distractions
\item Direct feedback
\item Continuously challenging
\end{itemize}

\end{frame}

%----------- slide --------------------------------------------------%
\begin{frame}
  \frametitle{Game Mechanics}

\begin{itemize}
\item Space
\item Objects, Attributes, and States
\item Actions
\item Rules
\item Skill
\item Chance

\end{itemize}

\end{frame}

%----------- slide --------------------------------------------------%
\begin{frame}
  \frametitle{Space}

\begin{itemize}
\item Discrete or continuous?
\item Number of dimensions?
\item Bounded or infinite?
\item Nested?
\item Mental spaces?
\end{itemize}

\end{frame}

%----------- slide --------------------------------------------------%
\begin{frame}
  \frametitle{Objects, Attributes, and States}


\begin{itemize}
\item Objects are state machines, their attributes depend on their states.
\item Secret attributes:  known, shared, etc.
\end{itemize}

\end{frame}

%----------- slide --------------------------------------------------%
\begin{frame}
  \frametitle{Actions}

\begin{itemize}
\item What are the {\em verbs}?
\item What actions are built in to the rules?
\item What objects can they act on?
\item What actions are {\em emergent}?
\item What side effects are there?
\end{itemize}

\end{frame}

%----------- slide --------------------------------------------------%
\begin{frame}
  \frametitle{Rules}

\begin{itemize}
\item What are the fundamental rules?
\item Are there ``laws'' or ``house rules''?
\item Are there modes in the game?
\item Who enforces the rules?
\item Are the rules easy to understand?

\end{itemize}

\end{frame}

%----------- slide --------------------------------------------------%
\begin{frame}
  \frametitle{The Most Important Rule of All}

\begin{itemize}
  \item What is the ultimate goal?
\item Is it clear to the players?
\item Is there a series of goals?  Do the players realize this?
\item Are the goals related to each other in a meaningful way?
\item Are the goals concrete, achievable, and rewarding?
\item Are short-term and long-term goals balanced?
\item Do players get to choose between goals?

\end{itemize}

\end{frame}

%----------- slide --------------------------------------------------%
\begin{frame}
  \frametitle{Skills}

\begin{itemize}
\item What skills are required?
\item Are there categories of skill missing?
\item Which skills dominate?
\item What experience do the skills create?
\item Are some players much better at these skills?  Does that make it
  unfair?  Is there a way to handicap the game?
\item Can players improve their skills?
\item Does this game demand the right amount of skill for its target
  audience? 

\end{itemize}

\end{frame}

%----------- slide --------------------------------------------------%
\begin{frame}
  \frametitle{Chance}

\begin{itemize}
\item Are certain events governed by chance?
\item What is the perceived chance?
\item What are the rewards?
\item Are the chance calculations complex?
\item Estimating chance is a {\em skill}.
\end{itemize}

\end{frame}

%----------- slide --------------------------------------------------%
\begin{frame}
  \frametitle{Balance}

\begin{itemize}
\item Fairness: 
 symmetrical: checkers, chess {\em vs.}
 asymmetrical: rock-paper-scissors, fox and geese
\pause
\item Challenge {\em vs.} Success: flow and the learning curve\pause
\item Meaningful Choices:  low risk/low reward {\em vs.} high
  risk/high reward \pause
\item Skill {\em vs.} Chance \pause
\item Head {\em vs.} Hands \pause
\item Competition {\em vs.} Cooperation \pause
\item Short {\em vs.} Long\pause
\item Rewards and Punishments Appropriate \pause
\item Freedom {\em vs.} Controlled Experience \pause
\item Simple {\em vs.} Complex: innate {\em vs>} emergent --- {\em elegance} \pause
\item Detail {\em vs.} Imagination\pause
\item Beware dynamic balance!
\end{itemize}

\end{frame}

%----------- slide --------------------------------------------------%
\begin{frame}
  \frametitle{Puzzles}

\begin{itemize}
  \item Goal easily understood
\item Easy to get started
\item Give a sense of progress
\item Give a sense of solvability
\item Increase difficulty gradually
\item Parallelism lets the player rest
\item Pyramid structure extends interest
\item Hints extend interest
\item Give the answer!
\item Perceptual shifts are a double-edged sword
\end{itemize}

\end{frame}


%----------- slide --------------------------------------------------%
\begin{frame}
  \frametitle{Interfaces}

\begin{itemize}
\item Does the game do what is expected when you use the control?
\item Do you get the experience of being in control?
\item Is the feedback too busy?
\item Does the feedback direct your attention correctly?
\item Is it a joy to use (swiffer {\em vs.} sweeping)?
\end{itemize}

\end{frame}



\end{document}

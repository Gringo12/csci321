\documentclass[t,compress]{beamer}
\usetheme{Singapore}
\usepackage{etex}

\sloppy
%\usepackage[scaled]{helvet}
%\usepackage{eulervm}

\usepackage{multicol}

\usepackage{fp-eval}
\usepackage{hyperref}
\usepackage{fancyvrb}
\usepackage{pstricks,pst-node,pst-tree,multido,pst-plot,pstricks-add}
\usepackage{graphicx}

\usepackage{alltt}


\newcommand{\bframe}[1]{\begin{frame}[fragile]{#1}}

\newcommand{\bbnf}{\begin{center}\begin{tabular}{rcl}}
\newcommand{\bnf}[2]{#1 & ::= & #2 \\}
\newcommand{\ebnf}{\end{tabular}\end{center}}

\newcommand{\myskip}{\vspace{-1em}\hrulefill}
\newcommand{\myrule}[1]{\vspace{-1ex}\centerline{\rule{#1cm}{1pt}}}
\newcommand{\myline}[1]{\centerline{#1}}
\newcommand{\bkt}[1]{\ensuremath{\langle\mbox{#1}\rangle}}
\newcommand{\br}{\mbox{~}|\mbox{~}}

\definecolor{orange}{rgb}{1,.5,0}
\definecolor{pink}{rgb}{1,.75,.75}
\definecolor{ltblue}{rgb}{.75,.75,1}

\newcommand{\be}{\begin{eqnarray*}}
\newcommand{\ee}{\end{eqnarray*}}
\newcommand{\bi}{\begin{itemize}}
\newcommand{\ei}{\end{itemize}}

\newcommand{\grph}[2]{\centerline{\includegraphics[scale=#1]{#2}}}

\AtBeginSection[]
{
\bframe{Outline}
\tableofcontents[currentsection]
\end{frame}
}

\title{Raven Notes 2}
\author{CSCI 321}
\institute{Based on {\em Programming Game AI by Example}, Buckland}

\begin{document}\small
\begin{centering}
\bframe{~}
\titlepage
\end{frame}

\bframe{Planning}
\bi
\item Involves more than one step at a time.
\item {\em Simple soccer} planning was absorbed in the logic:
  \bi 
  \item ``Move to support position'' 
    \bi
    \item not a goal in itself
    \item part of an underlying plan
    \ei
  \ei
\item {\em Raven} requires plans:
  \bi
  \item None of these is a simple action by itself:
    \bi
    \item Get health
    \item Get weapon
    \item Attack target
    \ei
  \ei
\ei
\end{frame}


\bframe{Hierarchical planning}


\flushleft $\bullet$ Buy sword \\

  \bi
  \item Get gold
    \bi 
    \item Plan path to goldmine
    \item Follow path
      \bi
      \item Follow edge \#1
      \item Follow edge \#2
      \item Follow edge \#3
      \ei
    \item Pick up nugget
    \ei
  \item Go to smithy
  \ei


Use an abstract class that can be either a simple or a 
composite goal.
\end{frame}


\bframe{Raven Goal Interface}

Similar to State interface

\bi
\item Activate
\bi\item can be called more than once to replan\ei
\item Process.  Returns one of
\bi
\item inactive
\item active
\item completed
\item failed
\ei
\item Terminate
\item HandleMessage
\item AddSubgoal

\ei

\end{frame}

\bframe{Raven Goals}

\centerline{\begin{tabular}{|l|l|}\hline
Composite goals & Atomic Goals \\\hline
Goal\_Think & Goal\_Wander \\
Goal\_GetItem   & Goal\_SeekToPosition    \\
Goal\_MoveToPosition   & Goal\_TraverseEdge    \\
Goal\_FollowPath   & Goal\_DodgeSideToSide    \\
Goal\_AttackTarget   & \\
Goal\_Explore   & \\
Goal\_HuntTarget   &\\\hline
\end{tabular}}
\end{frame}

\bframe{Atomic Steering Behavior Goals}

\bi
\item Goal\_Wander
\item Goal\_SeekToPosition
\item Goal\_TraverseEdge
\bi
\item upon activation checks for special edge behavior:
\bi\item open door\item swim\item {\em etc.}\ei
\item uses {\bf seek} for most edges, {\bf arrive} for last edge
\item monitors to check for stuck bot
\ei
\ei

\end{frame}

\bframe{Goal\_FollowPath}
\bi
\item Iterate through edges.
\item Edge type determines subgoal:
\bi
\item Goal\_TraverseEdge
\item Goal\_NegotiateDoor
\item Goal\_Jump
\item Goal\_Swim
\item {\em etc.}
\ei
\ei
\end{frame}

\bframe{Goal\_MoveToPosition}

\bi
\item Activate:
  \bi
  \item RequestPathToTarget (sent to Path Planner)
  \item SeekToPosition
  \ei
\item HandleMessage:
  \bi
  \item FollowPath (received from Path Planner)
  \ei
\ei
\end{frame}

\bframe{Goal\_AttackTarget}
\bi
\item If target gone:
  \bi \item Add subgoal: hunt target\ei
\item If target is shootable:
  \bi 
  \item If room:
    \bi
    \item Add subgoal: dodge side to side
    \ei
  \item else:
    \bi
    \item Add subgoal: seek to target position
    \ei
  \ei
\item Alternatives:
\bi\item move to best range for best weapon
\item move to sniping position
\ei
\item Weapon system is completely separate 
\bi\item  constantly 
  selects best weapon, aims, shoots
\item regardless of other goals
\ei
\ei

\end{frame}

\bframe{Goal\_Think}
\bi
\item Top level goal
\item Decides between:
  \bi
  \item Explore
\bi\item pick a random point and follow path there\ei
  \item Get Health
  \item Get Weapon
    \bi
    \item Rocket Launcher
    \item Shotgun
    \item Railgun
    \ei
  \item Attack Target
  \ei
  \item Uses four feature functions (all scaled 0-1):
    \bi
    \item Health
    \item Distance to item
    \item Individual weapon strength
    \item Total weapon strength
    \ei
\ei
\end{frame}

\bframe{Calculate Desirability and Choose Best}
\begin{eqnarray*}
Desirability(Health) &=& k\left(\frac{1-Health}{DistToHealth}\right)\\
Desirability(Weapon) &=& k\left(\frac{Health(1-WeaponStrength)}{DistToWeapon}\right)\\
Desirability_2(Weapon) &=& k\left(\frac{Health(1-WeaponStrength)}{DistToWeapon^2}\right)\\
Desirability(Attack) &=& k(TotalWeaponStrength)(Health)\\
Desirability(Explore) &=& 0.05
\end{eqnarray*}
\end{frame}

\bframe{Using Empathy}
\bi
\item You observe a player low on health break off a battle and run
\item You run your Goal\_Think algorithm on the player's data
\item The player's best option is to find health
\item You plan a path to the health to intercept the player
\ei
\end{frame}

\bframe{Personalities}
\bi
\item Desirability scores can be weighted.
\item Conservative player weighs health and weapons heavier than attack
\item Aggressive player weighs attack heavier
\item In a full RTS game you could:
  \bi
  \item Create opponent that favors exploration and research
  \item Create opponent that favors massive armies quickly
  \item Create opponent that favors city defenses
  \ei
\ei
\end{frame}

\bframe{State Memory}
\bi
\item Use a goal stack to resume interrupted goals.
\item Goal FollowPath could be interrupted by DefendAgainstAttacker
  and then resumed.
\item Goal FollowPath interrupted by NegotiateDoor and then resumed.
\ei
\end{frame}

\bframe{Command Queuing}
\bi
\item Used in modern RTS games
\item Can click many waypoints, Bot navigates to each in turn
\item Can establish patrols by making waypoints into a loop
\item Can queue multiple commands of any sort:
  \bi
  \item Build a barracks {\em and then}
  \item Move to this spot {\em and then}
  \item Build a turret
  \ei
\item Only change needed is adding subgoals to the back of the queue
  instead of the front.
\ei
\end{frame}

\bframe{Fuzzy Logic}
\bi
\item A more intuitive alternative to mathematical formulas for making
  decisions.  
\item Examples of fuzzy quantities:
\bi
\item A large piece of pie
\item A fairly strong wind
\item Low health
\item Hit the ball very hard
\item Far away
\ei
\ei


\end{frame}


\bframe{Crisp sets}

\psset{xunit=0.1cm,yunit=4cm}
\begin{pspicture}[showgrid=false](60,0)(140,1.25)
\psaxes[Dx=10,Ox=60]{->}(60,0)(140,1.25)
\listplot[linecolor=red]{70 0 70 1 90 1 90 0}
\listplot[linecolor=green]{90 0 90 1 110 1 110 0}
\listplot[linecolor=blue]{110 0 110 1 130 1 130 0}
\rput[b]{90}(60,0.5){Membership}
\rput[tl](140,0){IQ}
\rput[b](80,1.1){Dumb}
\rput[b](100,1.1){Average}
\rput[b](120,1.1){Clever}
\end{pspicture}

\end{frame}
\bframe{Fuzzy sets}

\psset{xunit=0.1cm,yunit=4cm}
\begin{pspicture}[showgrid=false](60,0)(140,1.25)
\psaxes[Dx=10,Ox=60]{->}(60,0)(140,1.25)
\listplot[linecolor=red]{60 0 75 1 85 1 95 0}
\listplot[linecolor=green]{85 0 95 1 105 1 115 0}
\listplot[linecolor=blue]{105 0 115 1 125 1 140 0}
\rput[b]{90}(60,0.5){Membership}
\rput[tl](140,0){IQ}
\rput[b](80,1.1){Dumb}
\rput[b](100,1.1){Average}
\rput[b](120,1.1){Clever}
\end{pspicture}

\end{frame}

\bframe{Fuzzy sets}

\psset{xunit=0.1cm,yunit=4cm}
\begin{pspicture}[showgrid=false](60,0)(140,1.25)
\psaxes[Dx=10,Ox=60]{->}(60,0)(140,1.25)
\listplot[linecolor=red]{60 0 80 1 100 0}
\listplot[linecolor=green]{ 80 0 100 1 120 0}
\listplot[linecolor=blue]{ 100 0 120 1 140 0}
\rput[b]{90}(60,0.5){Membership}
\rput[tl](140,0){IQ}
\rput[b](80,1.1){Dumb}
\rput[b](100,1.1){Average}
\rput[b](120,1.1){Clever}
\end{pspicture}

\end{frame}
\bframe{Average AND Clever}

\psset{xunit=0.1cm,yunit=4cm}
\begin{pspicture}[showgrid=false](60,0)(140,1.25)
\psaxes[Dx=10,Ox=60]{->}(60,0)(140,1.25)
\listplot[linecolor=lightgray]{60 0 80 1 100 0 80 0 100 1 120 0 100 0 120
  1 140 0}
\listplot{60 0 140 0}
\listplot{100 0 110 0.5 120 0}
\rput[b]{90}(60,0.5){Membership}
\rput[tl](140,0){IQ}
\rput[b](80,1.1){Dumb}
\rput[b](100,1.1){Average}
\rput[b](120,1.1){Clever}
\end{pspicture}

\end{frame}

\bframe{Average OR Clever}

\psset{xunit=0.1cm,yunit=4cm}
\begin{pspicture}[showgrid=false](60,0)(140,1.25)
\psaxes[Dx=10,Ox=60]{->}(60,0)(140,1.25)
\listplot[linecolor=lightgray]{60 0 80 1 100 0 80 0 100 1 120 0 100 0 120
  1 140 0}
\listplot{60 0 140 0}
\listplot{80 0 100 1 110 0.5 120 1 140 0}
\rput[b]{90}(60,0.5){Membership}
\rput[tl](140,0){IQ}
\rput[b](80,1.1){Dumb}
\rput[b](100,1.1){Average}
\rput[b](120,1.1){Clever}
\end{pspicture}

\end{frame}

\bframe{NOT Clever}

\psset{xunit=0.1cm,yunit=4cm}
\begin{pspicture}[showgrid=false](60,0)(140,1.25)
\psaxes[Dx=10,Ox=60]{->}(60,0)(140,1.25)
\listplot[linecolor=lightgray]{60 0 80 1 100 0 80 0 100 1 120 0 100 0 120
  1 140 0}
\listplot{60 0 140 0}
\listplot{60 1 100 1 120 0 140 1 145 1}
\rput[b]{90}(60,0.5){Membership}
\rput[tl](140,0){IQ}
\rput[b](80,1.1){Dumb}
\rput[b](100,1.1){Average}
\rput[b](120,1.1){Clever}
\end{pspicture}

\end{frame}


\bframe{Hedges}

\psset{xunit=1cm,yunit=4cm}
\begin{pspicture}[showgrid=false](-3,0)(3,1.25)
\psaxes[ticks=none,labels=none]{->}(-3,0)(3,1.25)
\rput[b]{90}(-3,0.5){Membership}
\rput(0.5,0.1){Very}
\rput(2,0.4){Fairly}
\listplot{-2.5 0.00193045413622771  -2.25 0.00632971542748575  -2 0.0183156388887342  -1.75 0.046770622383959  -1.5 0.105399224561864  -1.25 0.209611387151098  -1 0.367879441171442  -0.75 0.569782824730923  -0.5 0.778800783071405  -0.25 0.939413062813476  0 1  0.25 0.939413062813476  0.5 0.778800783071405  0.75 0.569782824730923  1 0.367879441171442  1.25 0.209611387151098  1.5 0.105399224561864  1.75 0.046770622383959  2 0.0183156388887342  2.25 0.00632971542748575  2.5 0.00193045413622771  }
\listplot[linestyle=dotted]{-2.5 0.0439369336234074  -2.25 0.0795595087182277  -2 0.135335283236613  -1.75 0.216265166829887  -1.5 0.32465246735835  -1.25 0.457833361771614  -1 0.606530659712633  -0.75 0.754839601989007  -0.5 0.882496902584595  -0.25 0.969233234476344  0 1  0.25 0.969233234476344  0.5 0.882496902584595  0.75 0.754839601989007  1 0.606530659712633  1.25 0.457833361771614  1.5 0.32465246735835  1.75 0.216265166829887  2 0.135335283236613  2.25 0.0795595087182277  2.5 0.0439369336234074  }
\listplot[linestyle=dotted]{-2.5 3.72665317207867e-06  -2.25 4.00652973929511e-05  -2 0.000335462627902512  -1.75 0.00218749111818288  -1.5 0.0111089965382423  -1.25 0.0439369336234074  -1 0.135335283236613  -0.75 0.32465246735835  -0.5 0.606530659712633  -0.25 0.882496902584595  0 1  0.25 0.882496902584595  0.5 0.606530659712633  0.75 0.32465246735835  1 0.135335283236613  1.25 0.0439369336234074  1.5 0.0111089965382423  1.75 0.00218749111818288  2 0.000335462627902512  2.25 4.00652973929511e-05  2.5 3.72665317207867e-06  }
\end{pspicture}

\vfill

Fairly = square root

Very = square
\vfill

\end{frame}

\bframe{Fuzzy Linguistic Variable}

\psset{xunit=0.1cm,yunit=4cm}
\begin{pspicture}[showgrid=false](60,0)(140,1.25)
\psaxes[Dx=10,Ox=60]{->}(60,0)(140,1.25)
\listplot[linecolor=red]{60 0 80 1 100 0}
\listplot[linecolor=green]{ 80 0 100 1 120 0}
\listplot[linecolor=blue]{ 100 0 120 1 140 0}
\rput[b]{90}(60,0.5){Membership}
\rput[tl](140,0){IQ}
\rput[b](80,1.1){Dumb}
\rput[b](100,1.1){Average}
\rput[b](120,1.1){Clever}
\end{pspicture}

\vfill

IQ = \{Dumb, Average, Clever\}

A set of fuzzy sets.

\vfill

\end{frame}

\bframe{Reasoning with FLV's}
\bi
\item Define FLV's for all quantities of interest:
  \bi
  \item Distance: close, medium, far
  \item Ammo: low, okay, loads
  \item Desirability: undesirable, desirable, very desirable
  \ei
\item Define fuzzy rules ({\em e.g} desirability of using rocket launcher):
  \bi
  \item If target is far and ammo is loads then desirable
  \item If target is close then undesirable
  \item {\em etc.}
  \ei
\item Start with crisp values (measurements) of distance and ammo
\item Fuzzify crisp values to fuzzy values
\item Reason with fuzzy values
\item Come to fuzzy conclusion
\item Defuzzify conclusion
\ei
\end{frame}

\bframe{Fuzzy Desirability}

\psset{xunit=0.1cm,yunit=4cm}
\begin{pspicture}[showgrid=false](0,0)(100,1.25)
\psaxes[Dx=10]{->}(0,0)(105,1.25)
\listplot{0 1 20 1 50 0 80 1 100 1}
\listplot{20 0 50 1 80 0}
\rput[b]{90}(0,0.5){Membership}
\rput[t](50,-0.25){Desirability}
\rput[b](10,1.1){Undesirable}
\rput[b](50,1.1){Desirable}
\rput[b](90,1.1){Very Desirable}
\end{pspicture}

\end{frame}


\bframe{Fuzzy Distance to Target}

\psset{xunit=0.025cm,yunit=4cm}
\begin{pspicture}[showgrid=false](0,0)(400,1.25)
\psaxes[Dx=50]{->}(0,0)(400,1.25)
\listplot{0 1 10 1 150 0 300 1 400 1}
\listplot{10 0 150 1 300 0}
\rput[b]{90}(0,0.5){Membership}
\rput[t](200,-0.25){Distance to Target}
\rput[bl](10,1.1){Close}
\rput[b](150,1.1){Medium}
\rput[b](350,1.1){Far}
\end{pspicture}

\end{frame}


\bframe{Fuzzy Ammo Status}

\psset{xunit=0.2cm,yunit=4cm}
\begin{pspicture}[showgrid=false](0,0)(50,1.25)
\psaxes[Dx=10]{->}(0,0)(50,1.25)
\listplot{0 1 15 0 40 1 50 1}
\listplot{0 0 15 1 40 0}
\rput[b]{90}(0,0.5){Membership}
\rput[t](15,-0.25){Ammo}
\rput[bl](0,1.1){Low}
\rput[b](15,1.1){Okay}
\rput[b](45,1.1){Loads}
\end{pspicture}

\end{frame}


\bframe{Fuzzy Rules for Desirability of Rocket Launcher}


\vfill

\begin{enumerate}
\item If target is far and ammo is loads then desirable
\item If target is far and ammo is okay then undesirable
\item If target is far and ammo is low then undesirable
\item If target is medium and ammo is loads then very desirable
\item If target is medium and ammo is okay then very desirable
\item If target is medium and ammo is low then desirable
\item If target is close and ammo is loads then undesirable
\item If target is close and ammo is okay then undesirable
\item If target is close and ammo is low then undesirable
\end{enumerate}
\vfill

\end{frame}

\bframe{Fuzzy Inference}
\begin{enumerate}
\item For each rule,
  \begin{enumerate}
  \item For each antecedant, calculate the degree of membership of the
    input data
  \item Calculate the rule's inferred conclusion based on these values
  \end{enumerate}
\item Combine all the inferred conclusions into a single fuzzy
  conclusion
\item Defuzzify the conclusion
\end{enumerate}
\end{frame}

\bframe{Example}
\bi
\item
Suppose target is at distance 200 and ammo is 8 rockets remaining.
\item
{\bf Rule one:}  If target is far and ammo is loads then desirable
\bi\item Distance 200 means target far has DOM 0.33
\item Ammo 8 means ammo loads has DOM 0.0
\item ANDing these together means desirable is 0.0
\ei\item
{\bf Rule two:} If target is far and ammo is okay then undesirable
\bi\item Distance 200 means target far has DOM 0.33
\item Ammo 8 means ammo okay has DOM 0.78
\item ANDing these together means undesirable is 0.33
\ei\item
{\bf Rule three:} If target is far and ammo is low then undesirable
\bi\item Distance 200 means target far has DOM 0.33
\item Ammo 8 means ammo low has DOM 0.2
\item ANDing these together means undesirable is 0.2
\ei\ei

\end{frame}

\bframe{FAM:  Fuzzy Associative Matrix}

\vfill

\begin{tabular}{c|c|c|c|}
 & Target close & Target medium & Target far \\\hline
Ammo low &\color{lightgray} Undesirable 0 & Desirable 0.2 & Undesirable 0.2 \\\hline
Ammo okay &\color{lightgray}  Undesirable 0 & VeryDesirable 0.67 & Undesirable 0.33 \\\hline
Ammo loads &\color{lightgray}  Undesirable 0 &\color{lightgray}  VeryDesirable 0 &\color{lightgray}  Desirable 0 \\\hline
\end{tabular}

\vfill
\begin{tabular}{c|c|}
Consequent & Confidence \\\hline
Undesirable & 0.33 \\\hline
Desirable & 0.2 \\\hline
VeryDesirable & 0.67 \\\hline
\end{tabular}
\vfill

\end{frame}

\bframe{Fuzzy Conclusion: Truncate each set}

\psset{xunit=0.1cm,yunit=4cm}
\begin{pspicture}[showgrid=false](0,0)(100,1.25)
\psaxes[Dx=10]{->}(0,0)(105,1.25)
\listplot[linecolor=red]{0 1 20 1 50 0}
\listplot[linecolor=red,linestyle=dashed]{0 0.33 45 0.33}
\listplot[linecolor=blue]{50 0 80 1 100 1}
\listplot[linecolor=blue,linestyle=dashed]{66 0.67 100 0.67}
\listplot[linecolor=green]{20 0 50 1 80 0}
\listplot[linecolor=green,linestyle=dashed]{20 0.2 80 0.2}
\color{black}
\rput[b]{90}(0,0.5){Membership}
\rput[t](50,-0.25){Desirability}
\rput[b](10,1.1){Undesirable}
\rput[b](50,1.1){Desirable}
\rput[b](90,1.1){Very Desirable}
\end{pspicture}

\end{frame}

\bframe{Fuzzy Conclusion: Truncate each set}

\psset{xunit=0.1cm,yunit=4cm}
\begin{pspicture}[showgrid=false](0,0)(100,1.25)
\psaxes[Dx=10]{->}(0,0)(105,1.25)
\listplot[linecolor=red]{0 0.33 40 0.33 50 0}
\listplot[linecolor=green]{20 0 25 0.2 75 0.2 80 0}
\listplot[linecolor=blue]{50 0 70 0.67 100 0.67}
\color{black}
\rput[b]{90}(0,0.5){Membership}
\rput[t](50,-0.25){Desirability}
\rput[b](10,1.1){Undesirable}
\rput[b](50,1.1){Desirable}
\rput[b](90,1.1){Very Desirable}
\end{pspicture}

\end{frame}

\bframe{Defuzzify:  Mean of Maximum}

\psset{xunit=0.1cm,yunit=4cm}
\begin{pspicture}[showgrid=false](0,0)(100,1.25)
\psaxes[Dx=10]{->}(0,0)(105,1.25)
\listplot[linecolor=red]{0 0.33 40 0.33 50 0}
\listplot[linecolor=green]{20 0 25 0.2 75 0.2 80 0}
\listplot[linecolor=blue]{50 0 70 0.67 100 0.67}
\color{black}
\rput[b]{90}(0,0.5){Membership}
\rput[t](50,-0.25){Desirability}
\rput[b](10,1.1){Undesirable}
\rput[b](50,1.1){Desirable}
\rput[b](90,1.1){Very Desirable}
\psline[arrows=->,arrowscale=2](85,0.75)(85,0)
\end{pspicture}

\end{frame}

\bframe{Defuzzify:  Centroid}

\psset{xunit=0.1cm,yunit=4cm}
\begin{pspicture}[showgrid=false](0,0)(100,1.25)
\psaxes[Dx=10]{->}(0,0)(105,1.25)
\listplot[linecolor=red]{0 0.33 40 0.33 50 0}
\listplot[linecolor=green]{20 0 25 0.2 75 0.2 80 0}
\listplot[linecolor=blue]{50 0 70 0.67 100 0.67}
\color{black}
\rput[b]{90}(0,0.5){Membership}
%\rput[t](50,-0.25){Desirability}
\rput[b](10,1.1){Undesirable}
\rput[b](50,1.1){Desirable}
\rput[b](90,1.1){Very Desirable}
\psline[arrows=->,arrowscale=2](62,0.75)(62,0)
\end{pspicture}

\vfill
\[ \sum_i\frac{i(membership_i)}{\sum_i membership_i}\]

\end{frame}

\bframe{Defuzzify: Average of Maxima}

\psset{xunit=0.1cm,yunit=4cm}
\begin{pspicture}[showgrid=false](0,0)(100,1.25)
\psaxes[Dx=10]{->}(0,0)(105,1.25)
\listplot[linecolor=red]{0 1 20 1 50 0}
\listplot[linecolor=red,linestyle=dashed]{0 0.33 45 0.33}
\psline[linecolor=red,arrows=->,arrowscale=2](10,1.05)(10,0)
\listplot[linecolor=green]{20 0 50 1 80 0}
\listplot[linecolor=green,linestyle=dashed]{20 0.2 80 0.2}
\psline[linecolor=green,arrows=->,arrowscale=2](50,1.05)(50,0)
\listplot[linecolor=blue]{50 0 80 1 100 1}
\listplot[linecolor=blue,linestyle=dashed]{66 0.67 100 0.67}
\psline[linecolor=blue,arrows=->,arrowscale=2](90,1.05)(90,0)
\color{black}
\rput[b]{90}(0,0.5){Membership}
%\rput[t](50,-0.25){Desirability}
\rput[b](10,1.1){Undesirable}
\rput[b](50,1.1){Desirable}
\rput[b](90,1.1){Very Desirable}
\end{pspicture}

\vfill

\[ \frac{\sum_i (value_i)(confidence_i)}{(\sum_i confidence_i)}\]

\end{frame}

\bframe{Combinatorial explosion of rules}

\vfill

If $A_i$ and $B_i$ and $C_i$ and $D_i$ then Consequent

\vfill

If each FLV has 5 possible values, then we have $5^4 = 625$ rules.

\vfill

If we had 6 LFV's instead of 4, there would be $5^6 = 15625$ rules.
\vfill


\end{frame}


\bframe{Rewrite rules for ease of computation}

If $A_i$ and $B_i$ and $C_i$ and $D_i$ then Consequent

\vfill

is equivalent to 

\vfill

If $A_i$ then Consequent\\OR\\
If $B_i$ then Consequent\\OR\\
If $C_i$ then Consequent\\OR\\
If $D_i$ then Consequent

\vfill

Using equivalences like this we can usually greatly reduce the number
of rules.  For example, the three rules for desirability of using the
rocket launcher when the target is close can all be reduced to one:
\vfill

If target is close then undesirable
\vfill

 This simplification can be done automatically.

\vfill

\end{frame}



\end{centering}

\end{document}

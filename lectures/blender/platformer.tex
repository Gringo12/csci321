\documentclass{article}
\usepackage[margin=1in]{geometry}
\title{Building a Simple Platformer with Blender}
\author{Geoffrey Matthews}
\newcommand{\bi}{\begin{enumerate}}
\newcommand{\ii}{\item}
\newcommand{\ei}{\end{enumerate}}
\newcommand{\arrow}{\ensuremath{\rightarrow}}

\newcommand{\but}[1]{\framebox{\sf #1}}

\begin{document}
\maketitle

\paragraph{Defaults}
\bi
\item Change the \but{Blender Render} dropdown at top to \but{Blender Game}.
\item Enlarge the right hand side panel.
\item In the Properties window, change the \but{Shading} from \but{
  Multitexture} to \but{Glsl}.
\item In the 3D window, change the \but{Viewport shading} from \but{
  Solid} to \but{Texture}.
\item  Enlarge the bottom panel and change the window from \but{
  Timeline} to \but{Logic editor}.
  \item Under the \but{Game} menu, enable \but{Show debug properties}.
\item You may now want to save your defaults, with \but{File \arrow User
  Preferences \arrow Interface \arrow Save User Settings}.

  If you do this,
  you only have to do the above steps once!
\ei

\paragraph{Character movement on a plane}
\bi
\item Drag the cube up a couple units.
\item Center the cursor with \but{Shift-C}
\item Add a mesh plane with \but{Shift-A}
\item Scale the plane with \but{S}, \but{20}, \but{Enter}.
\item Select the cube with \but{Right-click}.
\item In the physics button on the \but{Properties} panel, make the
  cube's \but{Physics type} a \but{Character}.
\item Also check the \but{Collision bounds} box.
\item In the 3d window, press \but{P} to play the game and see if the
  cube falls to the plane.
\item Press \but{Escape} to stop playing the game.
\item Select the cube with \but{Right-click}.
\item Add a \but{Sensor} of type \but{Keyboard}.
\item Click the \but{Key} field of the sensor and touch the \but{W}
  key.  Also name the sensor \but{W}
\item Add an \but{Actuator} of type \but{Motion}.
\item Change the motion from \but{Simple motion} to \but{Character motion}.
\item Change the \but{Y } field to \but{0.1}.
\item Wire up the sensor to the actuator.
\item In the 3d window press \but{P} and then in the game press \but{
  W} to see if the character moves.
\item Press \but{Escape} to stop playing the game.

\ei

\paragraph{Turning and jumping with the mouse}
\bi
\ii Add a \but{Mouse} sensor of type \but{Movement}
\ii Connect it to a \but{Mouse} actuator of type \but{Look}
\ii Disable the \but{Y} axis.
\ii Test your game, moving and turning.
\ii Add a \but{Keyboard} sensor for the \but{Space} key.
\ii Activate the \but{Tap} button on the sensor.
\ii Connect it to a \but{Movement } actuator of type \but{Character
  motion} and select the \but{Jump} button.
\ii Test your game, moving, turning, and jumping.
\ii If you want higher jumps, adjust the \but{Jump force} in the \but{
  Physics} button of the \but{Properties} panel.
\ei

\paragraph{Changing the shape of the cube}
\bi
\ii Select the cube.
\ii Press \but{Tab} to enter edit mode.
\ii Select the top two vertices on the \but{Y} side of the cube.
\ii Scale them down a bit:  press \but{S} and then slide the mouse,
then click.
\ii Move them back a bit: press \but{G} and then \but{Shift-x} and
then slide the mouse, then click
\ii Rename the cube.  Call it \but{Bob}.
\ei

\paragraph{Add colors}
\bi
\ii Change the lamp to a sunlamp.  Select the lamp, and in
the \but{Data} button on the \but{Properties} panel select \but{
  Sun}.
\ii Select the ground.  In the \but{Material} button on the \but{
  Properties} panel add a new material.
\ii Click in the \but{Diffuse} box and pick a color.
\ii Slide the \but{Specular} intensity down to zero.
\ii Similarly, pick a color for \but{Bob}.
\ii Adjust your lamp intensity as desired.
\ii You may want to add a (shadowless) backlight in the opposite
direction of your sun.  Turn it way down.
\ei

\paragraph{Better Cameras}
\begin{enumerate}
  \item Camera 1:
\bi
\ii Select the camera.
\ii Add an \but{Always} sensor.
\ii Add a \but{Camera} actuator.  Pick some values for
\but{Min}, \but{Max}, \but{Height} and \but{
  Damping}.
\ii Test the game,  experimenting
with the \but{Min}, \but{Max}, \but{Height} and \but{
  Damping} until you get it right.
\ei
\item Camera 2:
  \bi
\ii Add another camera.
\ii Position this camera behind \but{Bob} so you can see over his/her
shoulder:  In the \but{View} part of the \but{N-toggled} shelf in the
3d view, click \but{Lock Camera to View}.  Then fiddle with the camera
until you're happy.
\ii Unclick the \but{Lock Camera to View} and move to where you can
see both camera and Bob.
\ii Parent the camera to Bob:
\bi
  \ii Select the camera
  \ii Shift-select Bob
  \ii Press \but{Control-P} and parent to the \but{Object}.
\ei
\ii Test game.
\ii Select the camera by itself.
\ii Add another \but{Mouse} \but{Movement} sensor.
\ii Link it to a \but{Mouse} \but{Look} actuator.
\ii Disable the \but{X} axis in the actuator.
\ii Test game.
\ei
\ii Which camera do you like best?
\bi
\ii You can add a keypress to switch between cameras.
\ei
\ei

\paragraph{Add platforms}
\bi
\ii Position 3d cursor.
\ii Add a cylinder with \but{Shift-A}.  Name it \but{Platform}.
\ii Scale \but{Z} small and scale \but{X} and \but{Y} large.
Use \but{Shift-Z} to scale all {\em but} \but{Z}.
\ii Add a material, color, turn down the specular.
\ii Duplicate the platform and spread them about jumpable distances
from each other.
\ii Test the game.  Make one platform hard to get to.
\ei

\paragraph{Add moving obstacle}
\bi
\ii Add a cube just above one of the platforms.  Add material and
color. Name it \but{Blocker}.
\ii Add a \but{Property}:  Open the \but{N}-shelf in the \but{Logic} window,
click the plus.  Set the type of the property to \but{Integer}.  Set
its name to \but{Steps}.
\ii Enable debugging for \but{Steps} by clicking the little \but{i}.
\ii Add an \but{Always} sensor, and click the \but{True} pulse button
(three dots on top).
\ii Add a \but{Motion} actuator, moving \but{X} positive 0.1.
\ii Add a \but{Property } actuator, which adds 1 to \but{Steps}.
\ii Connect the \but{Always} sensor to {\em both} the \but{Motion} and
\but{Property } actuators using a single \but{And} controller.
\ii Add a \but{Property} actuator, which checks whether the \but{
  Steps} value is greater than 120 (two seconds).
\ii Connect this up to a \but{State} actuator which will change state
to the lower-left corner state.  (Actually state 16.)
\ii Switch to \but{State 16}.
\ii Add an \but{Always} sensor (enable the \but{True} pulse button)
that will move -0.1 in \but{X} every
step, and add -1 to the \but{Steps} property.
\ii Add a \but{Property} sensor that will change back to \but{State 1}
when \but{Steps} becomes less than 0.
\ii Test your game and see if the cube moves back and forth.
\ii Duplicate the cube, rotate it, and place copies over other
platforms. 
\ei

\paragraph{Ending the game}
\bi
\ii Select your ending platform.
\ii Add a \but{Collision} sensor.  Set it to detect collisions with
the \but{Material} of \but{Bob}.
\ii Connect this sensor to a \but{Scene} actuator that \but{Restart}s
the scene.
\ii Test your game.
\ei

\paragraph{Game over delay}
\bi
\ii Select your ending platform.
\ii Delete the \but{Scene} actuator that \but{Restart}s the game.
\ii Replace it with a \but{State}  actuator that sets the state to the
one in the lower left-hand corner.
\ii Reveal the \but{State} panel using the tiny plus sign next to the
\but{Add controller} in the logic panel.
\ii Select the state in the lower left-hand corner.
\ii Add a \but{Delay} actuator, with a delay of
360 (3 seconds).
\ii Connect the \but{Delay} actuator to a \but{Scene } actuator that
will \but{Restart} the game.
\ii Test your game.
\ei

\paragraph{Game Over overlay}
\bi
\ii Name the current scene \but{Game} by clicking inside the \but{
  Scene } drop-down at top.
\ii Add a new scene with the plus sign, and name it \but{Game Over
  overlay}.
\ii In this scene add a text object.  Edit the text to say \but{Game
  Over}.
\ii Convert the text to a mesh with the 3d view menu \but{
  Object \arrow Convert to \arrow Mesh}.
\ii Add a material and colorize the text however you like.  Edit the
letters, scale them, go nuts.  But leave it legible.
\ii Add a camera. 
\ii Change the camera to an \but{ Orthographic} camera  with the \but{
  Data} buttons in the \but{Properties} panel.
\ii Position the camera so you can see the text (use the \but{Lock
  Camera to View} checkbox.
\ii Go back to the \but{Game} scene and select the final platform.
\ii In the \but{Initial} state, where the platform has a \but{
Collision} with Bob event, add a \but{Scene} actuator that will \but{
Add Overlay Scene} with the \but{Game Overlay} scene
selected.
\ii In the other (lower-left) state, where the platform has a \but{
  Delay} sensor, add a \but{Scene } actuator that will \but{Remove
  Scene} with the \but{Game Overlay} scene selected.
\ii Test the game.
\ei

\end{document}



  

  
